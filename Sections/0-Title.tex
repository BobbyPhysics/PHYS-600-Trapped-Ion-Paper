\title{An Exploration of 2-D Coulomb Crystals in Quantum Computing}% Force line breaks with \\
%\thanks{A footnote to the article title}%

\author{Robert T. Hackett}
 %\altaffiliation[Also at ]{Physics Department, XYZ University.}%Lines break automatically or can be forced with \\
%\author{Second Author}%
 %\email{Second.Author@institution.edu}
\affiliation{%
 University of Washington,  Department of Physics\\
 Box 351560, Seattle, WA 98195-1560
}%

%\collaboration{MUSO Collaboration}%\noaffiliation

% \author{Charlie Author}
%  \homepage{http://www.Second.institution.edu/~Charlie.Author}
% \affiliation{
%  Second institution and/or address\\
%  This line break forced% with \\
% }%
% \affiliation{
%  Third institution, the second for Charlie Author
% }%
% \author{Delta Author}
% \affiliation{%
%  Authors' institution and/or address\\
%  This line break forced with \textbackslash\textbackslash
% }%

% \collaboration{CLEO Collaboration}%\noaffiliation

\date{\today}% It is always \today, today,
             %  but any date may be explicitly specified

\begin{abstract}
In the nearly 20 years since Cirac and Zoller demonstrated that laser-cooled ions confined to a linear trap could be used to implement quantum gates, ion trapping has proven to be a suitable platform for practical quantum computation \cite{Cirac}. It didn't take long to show that each of the main DiVincenzo criteria (excluding those concerning flying qubits) can be satisfied by trapped ion systems, including single-qubit and multi-qubit gates \cite{Bruzewicz}. Long trap lifetimes and internal-state coherence, along with strong interactions between ions, make it possible to achieve control and measurement with extremely high fidelity. At this point in time, the largest hurdle facing trapped ion quantum computing (and most other paradigms) is scalability, with some of the most advanced models being comprised of tens of qubits \cite{Moses}. In this paper, I will explore the use of 2-D Coulomb crystals of trapped ions for increasing the scale of future quantum processors \cite{Kato, Kiesenhofer}.
\begin{description}
\item[Keywords]
trapped ion, Coulomb crystals, quantum computing
\end{description}
\end{abstract}

\maketitle