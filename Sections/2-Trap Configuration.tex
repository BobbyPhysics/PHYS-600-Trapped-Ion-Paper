\section{Trapping Configurations}
Now that we have a better understanding of the types of ions used in trapped ion quantum computing, as well as some of the advantages and disadvantages they each have, we can turn our attention to the trapping configurations. 

When considering the possible geometries that an ion trap can have, we're essentially given two choices: one-dimensional (1D), also known as linear arrays; or two-dimensional (2D), sometimes called planar arrays, which is the basis for the 2D Coulomb crystals we're building up to. 

\subsection{Linear Arrays}
It may come as no surprise that the first quantum computers to be described using trapped ions made use of linear traps. Despite the simplistic geometry, all the necessary features of a quantum computer can be realized with linear ion traps. It's even possible to implement quantum gates between any set of ions in this scheme, not necessarily neighboring ions, for gates involving pairs, triplets, or any arbitrary number of ions. However, in practice, increasing the number of effective qubits is not an arbitrary task. Decoherence due to the environment interacting with the quantum system remains a significant challenge \cite{Cirac}.

Nevertheless, advances have been made that show how quantum computers based on linear arrays could provide solutions to difficult problems in materials design and molecular modeling through quantum simulation. In one example from 2017, Zhang \textit{et al.} successfully performed a quantum simulation of a dynamical phase transition (DPT) using up to 53 qubits in a linear ion trap. In this system, the qubits are coupled at long-range through their collective quantized motion due to Coulomb interactions. Each individual qubit is measured by a global long-range Ising interaction which has an efficiency of almost 99\%. This high efficiency makes it possible to measure many-body correlations between qubits in one shot, thereby allowing the DPT to be probed directly \cite{Zhang}.

The method employed by Zhang \textit{et al.} for confinement of long ion chains relied on a three-layer linear Paul trap with \ion{171}{Yb}{+} ions. Across the chain, ion spacing is anisotropic, ranging from \SI{1.5}{\micro\meter} at the center to \SI{3.5}{\micro\meter} towards the ends. As one might expect, the average lifetime of the chain scales inversely with the number of ions. At the maximum, 53 ions, an average lifetime of about 5 minutes was observed (which was sufficient for this experiment). The greatest factor in limiting the effective liftetime of the ion chain was Langevin collisions with residual background gas. When looking to scale up a system of this type, it will be crucial to implement cryogenic trap systems to reduce the pressure and collision energies \cite{Zhang}. 

More recently, a flexible scheme for maintaining efficient entanglement between ions in a long chain as the size scales up has been introduced by Leung and Brown. By utilizing both amplitude and frequency modulation, they were able to apply high-fidelity pulse sequences to drive transverse motional modes, which can suppress gate errors. However, this comes at the cost of gate power for improved robustness against errors \cite{Leung}.

\subsection{Planar Arrays}
This is a test of the github branch.

Unidentified vessel travelling at sub warp speed, bearing 235.7. Fluctuations in energy readings from it, Captain. All transporters off. A strange set-up, but I'd say the graviton generator is depolarized. The dark colourings of the scrapes are the leavings of natural rubber, a type of non-conductive sole used by researchers experimenting with electricity. The molecules must have been partly de-phased by the anyon beam.

Run a manual sweep of anomalous airborne or electromagnetic readings. Radiation levels in our atmosphere have increased by 3,000 percent. Electromagnetic and subspace wave fronts approaching synchronization. What is the strength of the ship's deflector shields at maximum output? The wormhole's size and short period would make this a local phenomenon. Do you have sufficient data to compile a holographic simulation?

I have reset the sensors to scan for frequencies outside the usual range. By emitting harmonic vibrations to shatter the lattices. We will monitor and adjust the frequency of the resonators. He has this ability of instantly interpreting and extrapolating any verbal communication he hears. It may be due to the envelope over the structure, causing hydrogen-carbon helix patterns throughout. I'm comparing the molecular integrity of that bubble against our phasers.

We're acquainted with the wormhole phenomenon, but this... Is a remarkable piece of bio-electronic engineering by which I see much of the EM spectrum ranging from heat and infrared through radio waves, et cetera, and forgive me if I've said and listened to this a thousand times. This planet's interior heat provides an abundance of geothermal energy. We need to neutralize the homing signal.

Communication is not possible. The shuttle has no power. Using the gravitational pull of a star to slingshot back in time? We are going to Starbase Montgomery for Engineering consultations prompted by minor read-out anomalies. Probes have recorded unusual levels of geological activity in all five planetary systems. Assemble a team. Look at records of the Drema quadrant. Would these scans detect artificial transmissions as well as natural signals?

Exceeding reaction chamber thermal limit. We have begun power-supply calibration. Force fields have been established on all turbo lifts and crawlways. Computer, run a level-two diagnostic on warp-drive systems. Antimatter containment positive. Warp drive within normal parameters. I read an ion trail characteristic of a freighter escape pod. The bomb had a molecular-decay detonator. Detecting some unusual fluctuations in subspace frequencies.

Resistance is futile.